\documentclass[12pt]{article}

\usepackage{amsmath,amssymb,amsthm}
\usepackage[T1]{fontenc}
\usepackage{graphics}
\usepackage{qtree}
\usepackage{tikz}
\usepackage[utf8]{inputenc} % æøå
\usepackage[T1]{fontenc} % mere æøå
\usepackage[danish]{babel} % orddeling
\usepackage{verbatim} % så man kan skrive ren tekst
\usepackage[all]{xy} % den sidste (avancerede) formel i dokumentet

\title{PKSU - Props 2.0} 
\author{Louise Knudsen}
\date{\today}

\newcommand{\R}{\mathbb{R}}
\newcommand{\C}{\mathbb{C}}
\newcommand{\N}{\mathbb{N}}
\newcommand{\Z}{\mathbb{Z}}
\newcommand{\Q}{\mathbb{Q}}

\newcommand{\og}{\wedge}

\begin{document}
\maketitle
\subsection*{13/03/2014 - 1. meeting}
We had our first meeting with the chairman of The Royal Danish Theatre's (RDT) props department, Mikkel Rasmus Theut, on March 13th 2014. The meeting was held at Mikkel's office from 15.30 to 17.30. \\
At the meeting we were introduced to the current database-system, \textit{Props}, which is developed by an employee in the props department, Claus Nepper Fakkenberg, who is soon retiring. Mikkel shows us how he (and the employees) use the system, which is only accessible through the theatre's desktop computers, as it is "installed/saved"???? on the I-drive(????) \\
The RDT has at least 5 different platforms they work on: \\
\textit{Props}, their prop database-system \\
\textit{DKT-net}, their intranet \\
\textit{kglteater.dk}, their external web-page  \\
\textit{Kongeplanen}, their planning and scheduling site \\
\textit{Cumulus}, their media database-system. \\
In the current state, these systems do not interact in any significant way. We discuss the possibility of incorporating the future database-system (for the time being referred to as \textit{Props 2.0}) in \textit{DKT-net}, as this would solve the security problem, due to the already existing login-system. However, the idea is downgraded, as Mikkel points out that \textit{DKT-net} is only used to an minimal extent and a wish to have some sort of guest login to \textit{Props 2.0} is of higher priority, and this would result in a conflict with the intranet-login. \\

\subsubsection*{Ønsker}
Der ønskes et system hvor det er muligt at lave en mere advanceret søgning - der foreslåes af Mikkel en form for træstruktur som bliver "større og større" i takt med at søgningen bliver mere og mere detaljeret. \\
En form for "ledigheds" indikation - der forslåes af Mikkel farver: GRØN = ledig rekvisit (ligger på lageret). ORANGE = optaget, men ikke i brug (ligger i container el.lign.). RØD = i brug (i repertoire).
Udover det ønskes muligheden for at reservere rekvisitter (blå?).\\
Der ønskes et web-baseret system, så det er muligt at logge ind på bærbare enheder. Der foreslåes fra IT-afdelingen et system lavet med mySQL og PHP (dette uddybes forhåbentlig på næste møde) - vi er mere interesseret i Python + framework (Bottle el.lign.). \\
Vi snakker om server-plads - Mikkel pointerer at vi bare skal sige til hvis vi skal bruge penge til udviklingen af systemet - dem har de massere af. \\\\
Vi aftaler at vi afholder næste møde torsdag d. 20/03/2014, hvor Martin fra IT-afdelingen vil være tilstede. Vi aftaler ligeledes at vi på et tidspunkt skal have et møde med møbelafdelingen - Palle og Thomas, som kan fortælle os mere om Møbelbogen, og muligvis Charlotte, som også har en finger med i spillet mht. den mystiske Møbelbog. 

\subsection*{20/03/2014 - 2. møde}
Torsdag d. 20/03/2014 afholdte vi møde med Mikkel, fra rekvisitten og Martin, fra IT-afdelingen. Martin fortæller os lidt om hvordan deres forskellige IT-systemer fungerer: \\ Kongeplanens data er på en Microsoft (SQL?) server, som det på sigt kunne være fedt hvis vores system automatisk kunne hente informationer fra om produktionerne for kommende sæsoner, så disse ikke skal indtastes manuelt (som Mikkel gør, som det ser ud nu).
\\
Cumulus er et program, som indekserer alle DKT's medier (billeder, videoer, lyd), som ligger seperat (somewhere) som tiff-filer, og Cumulus laver dem til jpeg-filer. Martin ser en mulighed i at ligge billeder af rekvisitterne ind i Cumulus, og så lave 'et spejl' af den meta data i mySQL, så vores system kan sammenkoble produktionerne med billederne ('tagge' billederne med forestillings-nr.?). Dette vil Martin stå for, med hans ord: Han klargører 'spejlet' så vi kan 'lyne' vores system på. \\\\
Vi igen om muligheden for gæsteadgang til det nye system, og når frem til at en gæstebruger højst sandsynligt kun vil være interesseret i billeder af rekvisitterne. Martin mener at dette kan gøres med Cumulus, så vi skrotter idéen om gæsteadgang. \\
Vi aftaler at vi koder systemet i PHP, selvom vores ønske var Python, men Martin pointerer at de fleste andre applikationer de har, er lavet med mySQL og PHP. Det er ligeledes det sprog Martin er bedst bekendt med, og vi bliver derfor enige om at det er mest hensigtsmæssigt at gøre sådan, hvilket muligvis også spare os noget support tid i sidste ende. \\\\
Vi afslutter mødet på Martins kontor, hvor vi får en USB pind med en sandkasse-version af Props, plus nogle andre ting med, og bliver informeret om at de på DKT skal til at opgradere til Windows 7 og Internet Explorer 9 (tror jeg?), og aftaler at Mikkel sætter et møde op med Palle og Thomas fra møbelafdelingen tirsdag d. 25/03/2014 kl. 13. Vi aftaler ligeledes at vi mødes d. 9/4/2014 og forventningsafstemmer, og får 100 \% på plads hvad systemet skal/ikke skal kunne, så vi kan komme igang med at kode.
\end{document}
